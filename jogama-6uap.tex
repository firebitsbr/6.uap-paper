\documentclass[titlepage]{article}

\usepackage[margin=1.25in]{geometry}
\usepackage{listings}
\usepackage{enumerate}
\usepackage{amsmath}
\usepackage{color}
\usepackage{graphicx}
\usepackage{tikz}
\usepackage{setspace}

% \usetikzlibrary{arrows,backgrounds,calc,trees,hobby}

\lstdefinestyle{mystyle}{                                                       
    backgroundcolor=\color{backcolour},                                         
    commentstyle=\color{codegreen},                                             
    keywordstyle=\color{magenta},                                               
    numberstyle=\tiny\color{codegray},                                          
    stringstyle=\color{codepurple},                                             
    basicstyle=\footnotesize,                                                   
    breakatwhitespace=false,                                                    
    breaklines=true,                                                            
    captionpos=b,                                                               
    keepspaces=true,                                                            
    numbers=left,                                                               
    numbersep=5pt,                                                              
    showspaces=false,                                                           
    showstringspaces=false,                                                     
    showtabs=false,                                                             
    tabsize=2                                                                   
}

\title{Untitled}
\author{Jonathan Garcia-Mallen}
\date{??? August 2016}

\doublespacing % using package 'setspace'

\begin{document}
\lstset{language=Bash,
  numbers=left,
  stepnumber=3,    
  firstnumber=1,
  numberfirstline=true
}
\maketitle
\tableofcontents

\pagebreak

\section{Background } 
Duckietown (2.166) is a graduate class on advanced autonomy taught at MIT, first taught in Spring 2016. It is a hands-on, project-focused course focusing on self-driving vehicles and high-level autonomy. Its students work to solve the underlying problem of designing the Autonomous Robo-Taxis System for the (fictional) City of Duckietown. Its students are diverse, coming from multiple departments with different backgrounds. 
\\
With this diversity in mind, the first week \textit{\textbf{or two}} is dedicated to bringing everyone on the same page, and doling out the robo-taxis to be programmed: Duckiebots. A Raspberry Pi 2 is at the center of these machines \textit{(cite duck site)}. To program them, students learn to log in remotely from their laptops to the robot's Pi and launch programs the same way, or by sending a command directly from their laptops without logging into their robot. The students had no way of running a program on their duckiebots without using their laptops.
\\
Picture this scenario. A grad student is testing some new autonomy on their laptop. Their reseach advisor \textit{(is this the right word?)} happens to walk by on her way to a meeting and asks him ``how's your duckiebot doing?'' The student rushes to power on his robot taxi (named batmobile), wait for it to connect to the network, and rushes the following incantation into his laptop's terminal: \textit{(change 'dat-grad-student' to 'you'?)}

% ah thank ye Jesus for waking me up at 0614 here in NH,
% letting me exercises, bathe go to morning prayer, pray a little more, and then making time for me to write her. Te lo agradezco, sinceramente. 
\begin{lstlisting}
  dat-grad-student@duckietop4:~$ ssh batmobile
  ssh: Could not resolve hostname batmobile.local: Name or service not known
  dat-grad-student@duckietop4:~$ ping batmobile.local
  ping: unknown host batmobile.local
  dat-grad-student@duckietop4:~$ ping batmobile.local
  PING batmobile.local (18.314.159.265)
  64 bytes from localhost (18.314.159.265): icmp_seq=1 ttl=64 time=396.039 ms
  64 bytes from localhost (18.314.159.265): icmp_seq=2 ttl=64 time=F I X   T  H E S E   N U M B E R S
  ^C
  --- localhost ping statistics ---
  4 packets transmitted, 4 received, 0% packet loss, time 2999ms
  rtt min/avg/max/mdev = 0.022/0.029/0.039/0.008 ms
  dat-grad-student@duckietop4:~$ ssh batmobile
  .
  .
  .
  # F I L L   T H E   R E S T   O F   T H I S
  ubuntu@batmobile:~$ roslaunch duckietown dat-grad-students-demo.launch  veh:=batmobile
 \end{lstlisting}
 \begin{lstlisting}
  or he does the other thing where he still has to ssh in but still sees all  the above errors that come up whenever you're not thinking straight,  running commands you're just getting familiar with. 
 
  you know, the remote roslaunch
  dad-grad-student@duckietop4:~$ roslaunch duckietown
  dat-grad-students-demo.launch veh:=batmobile
\end{lstlisting}

This is clearly a worst-case scenario. It is not the only scenario. A well-planned demo for a barely-technical audience would demand questions such as "Why do you need a laptop, if this is an autonomous vehicle?" or "Is the code running on the robot, or your computer?" And in any case, a laptop and the corresponding WiFi network necessary is yet another potential point of failure. A laptop should not be necessary in order to begin an autnonmous routine on the duckiebot. This 6.UAP project remedies this. 
\section{Requirements and Design Goals}
The purpose of this project is to create a quick and easy means to start any ROS program on the duckiebot. There is a clear primary requirement: this system must let the user (researcher or student) start a program of their choosing on the duckiebot without using any device external to the duckiebot. The duckiebot has two inputs that may be considered: a Raspberry Pi Cam 2, and a Logitech Joystick controler. 

Three goals guided the fulfillment of this requirement. The system must be reliable. It cannot fail when the user is in front of an audience. It must be easy to use and require as little interaction as is possible. Users shouldn't have a hard time interfacing with it, or have to push more buttons than either rosrun method. Lastly, this implementation must be future-proof \textit{(is there a more formal word? )}. The duckietown software will soon have to migrate to a different version of ROS and Ubuntu. The utilities produced by this project must be usable even as ROS and Ubuntu change. 
\section{Existing technologies used here}
We \textit{<--pl person?} must start a program at an unexpected time. Using input directly to the Raspberry Pi. We use its joystick. Input from the joystick must always be monitored. This monitor, this daemon, must start up on the duckiebot by itself. 

To interface with the joystick, we included this python module from github. 
init.d initializes EVERYTHING on Ubuntu 14.04. But it's been marked for death.
Supervisord is a python package that can also initialize thingies, and is actively developed by \textbf{\textit{x qntty of devs on gith}}
We explain these further now. 
\subsubsection{js_linux.py}
Nearly everything is a file in UNIX. This includes inputs from the joystick. They are written to a file \texttt{/dev/input/js0}, which is opened and read as such in python:

% snippet showing read block from code

\begin{lstlisting}
  # Open the joystick device.
  fn = '/dev/input/js0'
  print('Opening %s...' % fn)
  jsdev = open(fn, 'rb')
\end{lstlisting}

This snippet is from js_linux.py. This one-file script interacts is based off of the standard linux c interface. Using C would have narrowed greatly how many people can improve this code, so python here we go. 

Plus, we got this code from github. For free!
\subsubsection{System V}
When the duckiebot is powered on, it must begin listening for joystick commands. This background listening program, this daemon, must be launched on system startup. Ubuntu 14. 04 supports two initialization (init) systems, called by the Linux kernel to start every other process necessary to run the operating system. They are Upstart \textit{textbf{cite something}} and System V (sysv). The directories /etc/init and /etc/init.d correspond to either, respectively. Both init systems have been marked for death. Their replacement, systemd, is not readily available for Ubuntu 14.04. This poses a problem for any future-proof implementation. but they could not be avoided. We proceeded with sysv, as it has much greater support than upstart. 

We briefly explain sysv's interface. (/etc/init.d/README) (/etc/init.d/skeleton).  skeleton is 160 lines. 
This is pretty much all we use:
 
\begin{lstlisting}
  # PATH should only include /usr/* if it runs after the mountnfs.sh script
  PATH=/sbin:/usr/sbin:/bin:/usr/bin
  DESC="Description of the service"
  NAME=daemonexecutablename
  DAEMON=/usr/sbin/$NAME
  DAEMON_ARGS="--options args"
  PIDFILE=/var/run/$NAME.pid
  SCRIPTNAME=/etc/init.d/$NAME
 
  # Exit if the package is not installed
  [ -x "$DAEMON" ] || exit 0
\end{lstlisting}

This init script is used to start supervisord. We use supervisord version 3, installed via pip \textbf{textit{are you sure about that?}} because it is not in the 
\subsubsection{supervisord}

\section{Implementation}
% the subsubsection will be the individual files / indivisible components. 
\subsection{Joystick Daemon}
\subsubsection{joystick-daemon.py; startuptest}
\subsubsection{start-response.sh}
\subsection{Reliable startup script} % could be renamed? 
% \subsubsection{(runlevels, if need more padding)}
\subsubsection{init.d}
\subsubsection{rc.local}
\subsubsection{supervisord}
\section{Conclusions and Future work}
CONCLUSIONS
- results
- yo. it ran like 10 times in a row, successfully. It be good.
- It shouldn't have taken this long. It's not worth figuring out why
  rc.local didn't work, though. Who knows why.
FUTURE WORK. 
- Make it prettier. 
- once we switch to ubuntu 16.04, we should try to remove the dependency of supervisord. It's just not a good idea to depend on another thing. For now, it is a bridge to smooth such a transition. 
- Make it more user friendly. 
- we use ROS, arguably the most popular robotics middleware around. That didn't matter much at all for this project. This entire system could be used for a system running on MOOS, used by LAMSS, or LCM, used by the Robot Locomotion group. So long as a joystick is being used as input, the only file that would change would be start-response.sh.
\subsection{Acknowledgements}
\begin{itemize}
\item John Leonard, CEO(???) of Duckietown Engineering Co., for generously advising this and many others of my works
\item Liam Paull, COO of Duckietown Engineering Co., for advising me directly and patiently
\item Alex Chernovsky, of SIPB, for recommending supervisord
\item Anders, of pika, for giving me advice, though I forgot that advice
\item that person from office of EECS undergrads, for helping me get an incomplete when jleonard was busy
\item Kelly Shen, for planning advice, writing examples, and prayers. 
\end{itemize}

\subsection{References}
\begin{enumerate}[(1)]
\item https://www.kernel.org/doc/Documentation/input/joystick-api.txt
\item rdb had a nice gist on github.com
\end{enumerate}

\end{document}



%  LocalWords:  systemd sysv
